\begin{resumo}
GGovernan�a de Tecnologia da Informa��o (TI) tem-se tornado importante nas organiza��es, pois facilita a tomada de decis�es por meio do alinhamento do TI com a �rea (de neg�cio) foco da organiza��o. No entanto, a governan�a de TI � um processo lento e caro em termos de infraestrutura e recursos. Este trabalho apresenta um estudo sobre a governan�a de TI em um campus avan�ado da Universidade Federal de Santa Maria (UFSM-FW) e prop�em uma reorganiza��o dos servi�os de TI com suporte a ferramentas de c�digo aberto. O estudo demonstrou a necessidade de uma reestrutura��o do setor de TI e que o apoio de ferramentas para a automa��o pode tornar a UFSM mais sustent�vel.


PALAVRAS-CHAVE: Governan�a de TI, framework, feramenta.
%\ldots
\end{resumo}