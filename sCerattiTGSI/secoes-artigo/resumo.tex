\begin{resumo}
Governan�a de TI vem ampliando sua import�ncia com o passar do tempo sendo 
que fornece uma maior vis�o do que acontece em uma organiza��o fornecendo 
assim possibilidades para tomada de decis�es, dessa forma o estudo realizado 
sobre Governan�a de TI na Universidade Federal de Santa Maria campus de Frederico 
Westphalen, teve o intuito de analisar ferramentas existentes no campus, verificando 
que � usada um ferramenta de Help Desk utilizado pelo setor de TI, usada apenas para 
controle interno do setor, sendo analisado tamb�m uma ferramenta disponibilizada pelo 
software p�blico que pode ser usada de uma forma din�mica em todo o campus podendo suprir 
as necessidades do mesmo, a CITSMART possui uma ampla cobertura de v�rios setores podendo 
ser personalizada conforme necessidade da organiza��o, na ferramenta foi feito simula��es 
com os dados coletados no campus onde �  poss�vel observar o potencial da CITMART. Para 
fundamentar a pesquisa foi realizado um embasamento te�rico sobre Governan�a de TI e os 
frameworks mais usados como ITIL COBIT e ISO/IEC38500 para assim relacionar
com o estudo de caso em quest�o.
PALAVRAS-CHAVE: Governan�a de TI, framework, feramenta.
%\ldots
\end{resumo}