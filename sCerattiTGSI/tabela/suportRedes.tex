\begin{table}[H]
\tiny
	\centering
	
	\label{suport}
	\begin{tabular}{|l|l|}
		\hline
		Nome do Servi�o&Suporte de Redes \\
		\hline
		Status Atual do Servi�o&O servi�o � realizado pela equipe de TI da universidade, equipe esta composta de tr�s colaboradores efetivos e dois bolsistas. \\
		\hline
		Tipo de Servi�o&Suporte de Redes, atendimento na manuten��o e corre��o de erros ocorridos,  na rede do campus \\
		\hline
		Clientes&Toda comunidade acad�mica: Professores, t�cnicos, dire��o, alunos. \\
		\hline
		Descri��o do Servi�o&O servi�o � realizado sempre que solicitado, onde � realizado manuten��o de redes j� existentes, onde s�o corrigidos eventuais problemas. \\
		\hline
		Justificativa&Conseguir atender os clientes, para que as dificuldades enfrentadas sejam solucionadas, os dados encontrados geram feedback  \\
		\hline
		Resultados desejados &Satisfa��o do usu�rio, em rela��o a rede, para que assim possam realizar suas atividades   \\
		\hline
		Depend�ncia&O atendimento sofre depend�ncia da telefonia j� que os chamados s�o feitos atrav�s do telefone. \\
		\hline
		Mudan�as Planejadas para o servi�o&A falta de m�o de obra n�o possibilita a realiza��o de melhorias.  \\
		\hline
		Refer�ncias a planos pertinentes&Necessidade de aperfei�oamento, para efetuar melhorias. \\
		\hline
		Business Case&Riscos decorrentes a depend�ncia . \\
		\hline
	\end{tabular}
	%\label{tab:Nome_Da_Tabela}
	\caption{An�lise de Portif�lio: Suporte de Redes}
\end{table}