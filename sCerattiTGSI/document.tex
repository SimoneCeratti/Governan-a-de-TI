
%%% Versao 1.0
%%% Data: Novembro de 2015
%%% Autor: Simone

%%% Definicao da classe para a proposta e artigo no padrao SBC
\documentclass[12pt,a4paper]{article}

\usepackage{graphicx,url}
\usepackage{hyperref} 
\usepackage{authblk}
\usepackage{fancyhdr}
\usepackage[latin1]{inputenc}
\usepackage[brazil]{babel}   
\usepackage[T1]{fontenc}  
%\usepackage[utf8]{inputenc}
\usepackage{datetime}  
\usepackage[table,xcdraw]{xcolor}
\usepackage{hyperref}
\usepackage{float}
%\input{macros.tex} 
\usepackage{graphicx}
%\usepackage[affil-it]{authblk}

%%%------------------SBC-ARTIGO--------------------------
%%% Pacote com o padrao SBC
\usepackage{sbc-template}
%%%% Titulo, autores e identificacao no padrao SBC
\sloppy
%%% Colocar o titulo do seu trabalho
\title{Governan�a de Tecnologia da Informa��o: Um Estudo na Universidade Federal de Santa Maria}
%%% colocar o seu nome seguido do co-orientador (se existir) e orientador
\author{Simone Aparecida Ceratti, Cristiano Bertolini}

\address{Universidade Federal de Santa Maria - UFSM\\
 Centro de Educa��o Superior Norte - CESNORS, Frederico Westphalen, RS 
  \email{smone.ceratti@hotmail.com, cristiano.bertolini@ufsm.br}}
 
%%%------------------SBC-ARTIGO--------------------------

\begin{document}  

%%% Para o artigo vc deve incluir o comando abaixo
 \maketitle

%%% vc devera comentar ou a proposta ou artigo

%%%----------------------------------- BEGIN PROPOSTA
%\input{secoes-proposta/capa.tex}
%\input{secoes-proposta/dadosID.tex}
%\input{secoes-proposta/objetivos.tex}
%\input{secoes-proposta/motivacao.tex}
%%\begin{referencialTeorico}
\section{Governan�a de Tecnologia da Informa��o}
\label{sec:referencialTeorico}

Governan�a Corporativa � um ''sistema pelo qual organiza��es s�o dirigidas,
monitoradas e incentivadas, envolvendo todas as �reas da empresa com a �rea de
TI''Mancine \cite{Mancini}. A governan�a corporativa fundamenta-se em quatro
pilares: \textbf{Transpar�ncia}Transpar�ncia deixando transparecer as a��es
as parte interessadas; \textbf{Equidade}Equidade onde o tratamento � justo
a todos na organiza��o; \textbf{Presta��o de Contas}, prestando contas e
assumindo os resultados para suporte dos respons�veis pelo TI da organiza��o;
\textbf{Responsabilidade Corporativa} ,
zela pela organiza��o para que tenha longevidade e sustentabilidade Mancine
\cite{Mancini}.
	   
As organiza��es precisam melhorar o gerenciamento das informa��es que regem sua 
sobreviv�ncia no mercado, pois essas informa��es e tecnologias s�o fundamentais,
para garantir a integridade das mesmas. A partir dessa necessidade surgiram os modelos de Governan�a de TI para ajudar as organiza��es a gerirem suas tecnologias fornecendo 
ferramentas e m�tricas para o alinhamento entre os processos de TI e os objetivos 
estrat�gicos da organiza��o, Rodrigues \cite{R}.
	
\subsection{ISO/IEC38500}

Governan�a de TI auxilia na administra��o, integrando TI em todas as �reas da organiza��o, 
deixando de ser vista como gasto e sim 
como ajuda. ISO/IEC38500 � uma norma de alto n�vel, feita para diretores. Motiva
o uso da TI como ferramenta de trabalho, as organiza��es precisam da tecnologia para conseguir 
crescer, por�m seu custo � significativo e precisa ser usada de forma correta. 
A governan�a de TI envolve al�m de recursos f�sicos recurso humano, por essa raz�o est� 
sujeito a erros. A ISO/IEC38500 foi padronizada oferecendo um framework para governan�a 
efetiva de TI, definindo normas e princ�pios, diferenciando governan�a de gerencia, onde 
governan�a atinge todas as �reas da organiza��o e gerencia est� mais focada aos processos. 
Por�m, governan�a precisa de gerencia, para manter a organiza��o, ISO/IEC38500
\cite{iso}.
      
	Na norma ISO/IEC38500 a governan�a de TI � aplic�vel em 
	qualquer organiza��o de qualquer ramo e tamanho, por ser gen�rica, tendo com 
	principal objetivo o uso de TI. Nesta norma foram estabelecidos 6 princ�pios
	ISO/IEC38500 \cite{iso}:
	\begin{itemize}
     \item Responsabilidade: Define os respons�veis e pelo que ser�o respons�veis tendo 
     autonomia na tomada de decis�es. Aplicada para indiv�duos ou grupo de indiv�duos.
     \item Estrat�gia: Considera as capacidades atuais de TI, identificando as condi��es 
     da organiza��o, deve satisfazer as necessidades atuais e futuras, as estrat�gias tem 
     que estar alinhadas ao neg�cio.
     \item Aquisi��o: S�o feitas com base em an�lises apropriadas, com tomada de decis�o 
     clara, levando em conta custos, benef�cios e necessidades.
     \item Performance: TI satisfaz as necessidades da organiza��o, trazendo lucro, 
     na busca por atender requisitos do neg�cio, medindo resultados.
    \item Conformidade: estar em conformidade com as leis e regulamentos levando em 
    conta a legisla��o.
    \item Comportamento Humano: respeitar as individualidades de cada um.  
    \end{itemize}
   
    Os modelos de neg�cio devem seguir padr�es de como realizar as etapas da governan�a, 
    voltada para diretores das organiza��es baseadas em tr�s tarefas: Avaliar, os diretores 
    devem saber o estado atual da organiza��o, avaliando ambiente onde est� envolvido, 
    considerando a evolu��o continua, analisando necessidades atuais e futuras e qual a 
    vantagem competitiva. Direcionar, os diretores devem delegar atividades para os 
    respons�veis de cada �rea, gerenciando corretamente, estando em conformidade com a 
    infraestrutura existente e seus especialistas, incentivando a melhoria da cultura 
    organizacional. Monitorar, os diretores devem monitorar a TI usando m�tricas, verificando 
    a conformidade com a lei, usar tarefas dentro dos princ�pios, ISO/IEC38500 \cite{iso}.

     
     \subsection{COBIT(\emph{Control Objectives for Information and RelatedTechnology})}
     
     O COBIT � um guia para a gest�o das melhores pr�ticas da TI voltado para processos 
    e controles, utiliza um framework que fornece melhores pr�ticas para  gerenciamento 
    de processos de tecnologia da informa��o de uma forma estruturada, gerenci�vel e l�gica,
    idealizado para atender necessidades da Governan�a Corporativa, com foco nos requisitos
    de neg�cio, utilizando mecanismos de controle e an�lise de indicadores de desempenho, 
    poder� ser utilizado por qualquer empresa, independe das tecnologias empregadas na
    mesma, n�o importando se � de pequeno ou grande porte, segundo ISACA \cite{Cobit}. 
    COBIT est� dividido em quatro dom�nios:
     
     \begin{itemize} 
     \item Planejamento e Organiza��o: trata as estrat�gicas e t�ticas que TI 
     possa contribuir para realiza��o dos objetivos de neg�cio; 
     \item Aquisi��o e Implementa��o: determina a estrat�gia de TI para
      identificar, qualificar e escolher solu��es a serem desenvolvidas ou adquiridas; 
      \item Entrega e Suporte: verificar os servi�os requeridos pelos processos
      de neg�cio, para que haja entrega de informa��es e suporte para as opera��o em 
      situa��es inesperadas; 
       \item Monitora��o: monitorar os controles da organiza��o de TI, sendo fundamental 
       avalia��o continua e regular da qualidade e da conformidade dos controles implantados.
       \end{itemize}

 	Segundo ISACA \cite{Cobit} COBIT �  baseado em 5 princ�pios que permitem que a 
 	organiza��o construa um framework de governan�a e gest�o de TI baseado em um 
 	conjunto de 7 habilitadores que otimizam investimentos em tecnologia e informa��o 
 	utilizados para o benef�cio dos interessados:
  \begin{itemize} 
    \item Princ�pio 1. Atender as necessidades dos stakeholders: As organiza��es
    existem para criando valor para os envolvidos.
    \item Princ�pio 2. Cobrir a organiza��o de ponta a ponta: os gestores de neg�cio 
    t�m a responsabilidade de tratara TI como um ativo estrat�gico, onde TI � igual aos 
    demais.
    \item Princ�pio 3. Aplicar um framework �nico e integrado: integrar
    todos os conhecimentos existentes em diferentes frameworks.
    \item Princ�pio 4. Possibilitar uma abordagem hol�stica: O COBIT utiliza de
    7 viabilizadores(Princ�pios, Pol�ticas e Frameworks, Processos, Estruturas 
    Organizacionais, Cultura, �tica e Comportamento, Informa��o, Servi�os, 
    Infraestrutura e Aplica��es, Pessoas, Habilidades e Compet�ncias) que apoiam a 
    governan�a e a gest�o de TI abordar a organiza��o de ma forma completa.
    \item Princ�pio 5. Separar a governan�a da gest�o: Determinar onde cada �rea atuara, suas 
    estruturas e prop�sitos.
    \end{itemize} 
 
    
    \subsection{ITIL ( \emph{Information Technology Infrastructure Library})}
   
    ITIL � a jun��o das melhores pr�ticas para auxiliar organiza��es tanto no
    setor publico como no privado, usando as melhores formas para a cria��o execu��o 
    e manuten��o de servi�os, dessa forma entregar valores aos clientes, assim facilitar 
    os resultados desejados. BALDIN \cite{Baldin}
    
    A ITIL teve seu inicio nos anos 80 pelo CCTE (Central Computer and Tele
    communications Agency), para atender necessidade do governo brit�nico onde a 
    insatisfa��o com a TI era preocupante. 
    Assim, foi desenvolvido um conjunto de boas pr�ticas para gerenciar a utiliza��o 
    eficiente e respons�vel dos recursos de TI, independendo dos fornecedores sendo 
    aplic�vel a qualquer organiza��o respeitando as necessidades especificas de cada uma 
    delas. ITIL � um framework que descreve as melhores pr�ticas no gerenciamento de 
    servi�o de TI. ITIL fornece para a governan�a de TI um framework para gerenciamento 
    e controle de TI, focando no uso de m�tricas e melhoria da qualidade dos servi�os, 
    podendo fornecer benef�cios para os diretores, aumentando a satisfa��o dos usu�rios 
    e clientes, melhorando a tomada de decis�o e diminuindo os riscos FILHO \cite{ITIL}.

 


%\end{referencialTeorico}
%\input{secoes-proposta/solucaoProposta.tex}
%%% Caso vc nao tenha nenhum requisito especifico de Software ou hardware vc
%%% pode excluir a linha abaixo
%\input{secoes-proposta/requisitosSwHw.tex}
%\input{secoes-proposta/cronograma.tex}
%%%-----------------------------------%% END PROPOSTA 

%%%-----------------------------------%% BEGIN ARTIGO
 
  \begin{abstract}
Governance of information technology (IT) is becoming important in the organizations, 
because it facilitates decision-making through the alignment of IT with the area 
(of business) the focus of the organization. However, the IT governance is a slow 
and expensive process in terms of infrastructure and resources. This paper presents a 
study about IT governance in an advanced campus of the Federal University of Santa Maria 
(UFSM-FW) and propose a reorganization of the IT services with support to the open source 
tools. The study demonstrated the necessity of a restructuring of the IT sector and that 
the support of the tools for automation can turn the UFSM more manageable. 
KEY-WORDS: IT Governance, framework, tool.

%\ldots
\end{abstract}
  \begin{resumo}
Governan�a de Tecnologia da Informa��o (TI) tem-se tornado importante nas 
organiza��es, pois facilita a tomada de decis�es por meio do alinhamento do TI 
com a �rea (de neg�cio) foco da organiza��o. No entanto, a governan�a de TI � um 
processo lento e caro em termos de infraestrutura e recursos. Este trabalho apresenta 
um estudo sobre a governan�a de TI em um campus avan�ado da Universidade Federal de 
Santa Maria (UFSM-FW) e prop�em uma reorganiza��o dos servi�os de TI com suporte a 
ferramentas de c�digo aberto. O estudo demonstrou a necessidade de uma reestrutura��o 
do setor de TI e que o apoio de ferramentas para a automa��o pode tornar a UFSM mais 
gerenci�vel.

PALAVRAS-CHAVE: Governan�a, framework, ferramenta.
%\ldots
\end{resumo}
  
\section{Introdu��o}
\label{sec:introducao}
Governan�a de TI � instrumento de apoio importante a ser seguida pelas organiza��es, por 
auxilia na tomada de decis�es. Quando deparado com uma institui��o de ensino superior a 
amplitude da Governan�a de TI � incalcul�vel em se tratando de sua import�ncia, assim 
despertou o interesse de realizar estudos na institui��o relevantes a Governan�a de TI 
e descobrir oque realmente dela � usado, como � usado e em que pode ser melhorando e 
acrescentado. Governan�a de TI pode abrir horizontes, com novos caminhos a ser seguido para 
chegar ao objetivo desejado de forma eficiente e eficaz REFERENCIA. 

Governan�a de TI possibilita melhorar a estrutura de uma organiza��o possibilitando estudos 
e quando iniciado o estudo da mesma aumenta o interesse, e quando realizado a pesquisa em 
quest�o, identifica que uma das principais problem�ticas de Governan�a de TI � a falta de 
infraestrutura, pessoas preparadas e dispon�veis para tal fun��o e no caso da institui��o por 
a ferramenta usada no processo de Help Desk n�o ser institucionalizada, sendo usada somente 
para controle interno do setor de TI da universidade, sendo que essa � uma das poucas 
atividades de Governan�a de TI encontrada na universidade.

Governan�a de TI pode oferecer diversos benef�cios para as organiza��es, considera��o j� que 
uma organiza��o � um conjunto de fatores internos e externos que afetam a mesma diretamente. A 
Governan�a de TI auxiliara na tomada de decis�o de forma �gil, encontrando uma solu��o 
adequada em tempo h�bil. Ao analisar Governan�a de TI, verifica-se que um dos problemas no 
uso � a sua complexidade, pois exige tempo, investimento e recursos humanos. Por�m quando 
analisado os benef�cios que Governan�a de TI verifica-se que os gastos necess�rios s�o 
amplamente recuperados quando se tem dados relevantes para poder tomar decis�es. Governan�a 
de TI � motivada por: TI como  prestadora de servi�os, integra��o tecnol�gica, seguran�a da 
informa��o, depend�ncia do neg�cio em rela��o � TI, marcos de regula��o e ambiente de neg�cio
 ~ \cite{f}.

Governan�a de TI � o sistema em que o uso atual e futuro da TI s�o dirigidos e controlados, 
avaliados e direcionados para dar suporte � organiza��o e monitorar o uso de TI. Buscando 
integrar TI a todas as �reas da organiza��o, ISO/IEC 38500 \cite{iso}.

Este artigo prop�e uma analise e sugest�o de uso de Governan�a de TI fazendo uso de 
ferramentas de software livres, dessa forma amenizando os custos, assim viabilizaria a 
implanta��o de Governan�a De TI na universidade por ser um �rg�o p�blico.  
 
Neste contexto, o presente artigo est� organizado da seguinte forma: 
a se��o \ref{sec:referencialTeorico} apresenta um referencial te�rico, que
aborda algumas informa��es sobre Governan�a de TI  conceitualiza��o da mesma e 
informa��o sobre seus principais framework ISO/IEC38500,COBIT e ITIL. 
Na\ref{sec:roteiroDeImplantacao} est� exposto um roteiro de implanta��o com dez passos 
poss�veis para facilitar a implanta��o de Governan�a de TI em uma organiza��o. 
Na se��o \ref{sec:citismart} est� descrito o funcionamento da ferramenta de auxilio a 
Governan�a de TI, CITSMART, ferramenta esta dispon�vel no software p�blico de forma gratuita. 
Na se��o\ref{sec:modeloProposto}, est� exposta uma contextualiza��o de um modelo proposto 
de Governan�a de TI, com uma breve fundamenta��o de princ�pios da norma ISO/IEC38500, 
tamb�m um mapeamento de servi�os usando ITIL, definindo portf�lio de servi�os, cat�logo de 
servi�os, servi�o de Help Desk e gest�o de incid�ncias. Na se��o \ref{sec:analise} esta sendo 
descrito a analise do question�rio aplicada � comunidade acad�mica. Na se��o 
\ref{sec:trabalhosRelacionados} onde se encontra os trabalhos relacionados, 
foram citadas quatro pesquisas para acompanhamento em processos de implanta��o de 
Governan�a de TI.  Finalizando o artigo s�o apresentadas na se��o \ref{sec:conclusoes} as 
considera��es finais e finalizando as  refer�ncias empregadas.






  %\begin{referencialTeorico}
\section{Governan�a de Tecnologia da Informa��o}
\label{sec:referencialTeorico}

Governan�a Corporativa � um ''sistema pelo qual organiza��es s�o dirigidas,
monitoradas e incentivadas, envolvendo todas as �reas da empresa com a �rea de
TI''Mancine \cite{Mancini}. A governan�a corporativa fundamenta-se em quatro
pilares: \textbf{Transpar�ncia}Transpar�ncia deixando transparecer as a��es
as parte interessadas; \textbf{Equidade}Equidade onde o tratamento � justo
a todos na organiza��o; \textbf{Presta��o de Contas}, prestando contas e
assumindo os resultados para suporte dos respons�veis pelo TI da organiza��o;
\textbf{Responsabilidade Corporativa} ,
zela pela organiza��o para que tenha longevidade e sustentabilidade Mancine
\cite{Mancini}.
	   
As organiza��es precisam melhorar o gerenciamento das informa��es que regem sua 
sobreviv�ncia no mercado, pois essas informa��es e tecnologias s�o fundamentais,
para garantir a integridade das mesmas. A partir dessa necessidade surgiram os modelos de Governan�a de TI para ajudar as organiza��es a gerirem suas tecnologias fornecendo 
ferramentas e m�tricas para o alinhamento entre os processos de TI e os objetivos 
estrat�gicos da organiza��o, Rodrigues \cite{R}.
	
\subsection{ISO/IEC38500}

Governan�a de TI auxilia na administra��o, integrando TI em todas as �reas da organiza��o, 
deixando de ser vista como gasto e sim 
como ajuda. ISO/IEC38500 � uma norma de alto n�vel, feita para diretores. Motiva
o uso da TI como ferramenta de trabalho, as organiza��es precisam da tecnologia para conseguir 
crescer, por�m seu custo � significativo e precisa ser usada de forma correta. 
A governan�a de TI envolve al�m de recursos f�sicos recurso humano, por essa raz�o est� 
sujeito a erros. A ISO/IEC38500 foi padronizada oferecendo um framework para governan�a 
efetiva de TI, definindo normas e princ�pios, diferenciando governan�a de gerencia, onde 
governan�a atinge todas as �reas da organiza��o e gerencia est� mais focada aos processos. 
Por�m, governan�a precisa de gerencia, para manter a organiza��o, ISO/IEC38500
\cite{iso}.
      
	Na norma ISO/IEC38500 a governan�a de TI � aplic�vel em 
	qualquer organiza��o de qualquer ramo e tamanho, por ser gen�rica, tendo com 
	principal objetivo o uso de TI. Nesta norma foram estabelecidos 6 princ�pios
	ISO/IEC38500 \cite{iso}:
	\begin{itemize}
     \item Responsabilidade: Define os respons�veis e pelo que ser�o respons�veis tendo 
     autonomia na tomada de decis�es. Aplicada para indiv�duos ou grupo de indiv�duos.
     \item Estrat�gia: Considera as capacidades atuais de TI, identificando as condi��es 
     da organiza��o, deve satisfazer as necessidades atuais e futuras, as estrat�gias tem 
     que estar alinhadas ao neg�cio.
     \item Aquisi��o: S�o feitas com base em an�lises apropriadas, com tomada de decis�o 
     clara, levando em conta custos, benef�cios e necessidades.
     \item Performance: TI satisfaz as necessidades da organiza��o, trazendo lucro, 
     na busca por atender requisitos do neg�cio, medindo resultados.
    \item Conformidade: estar em conformidade com as leis e regulamentos levando em 
    conta a legisla��o.
    \item Comportamento Humano: respeitar as individualidades de cada um.  
    \end{itemize}
   
    Os modelos de neg�cio devem seguir padr�es de como realizar as etapas da governan�a, 
    voltada para diretores das organiza��es baseadas em tr�s tarefas: Avaliar, os diretores 
    devem saber o estado atual da organiza��o, avaliando ambiente onde est� envolvido, 
    considerando a evolu��o continua, analisando necessidades atuais e futuras e qual a 
    vantagem competitiva. Direcionar, os diretores devem delegar atividades para os 
    respons�veis de cada �rea, gerenciando corretamente, estando em conformidade com a 
    infraestrutura existente e seus especialistas, incentivando a melhoria da cultura 
    organizacional. Monitorar, os diretores devem monitorar a TI usando m�tricas, verificando 
    a conformidade com a lei, usar tarefas dentro dos princ�pios, ISO/IEC38500 \cite{iso}.

     
     \subsection{COBIT(\emph{Control Objectives for Information and RelatedTechnology})}
     
     O COBIT � um guia para a gest�o das melhores pr�ticas da TI voltado para processos 
    e controles, utiliza um framework que fornece melhores pr�ticas para  gerenciamento 
    de processos de tecnologia da informa��o de uma forma estruturada, gerenci�vel e l�gica,
    idealizado para atender necessidades da Governan�a Corporativa, com foco nos requisitos
    de neg�cio, utilizando mecanismos de controle e an�lise de indicadores de desempenho, 
    poder� ser utilizado por qualquer empresa, independe das tecnologias empregadas na
    mesma, n�o importando se � de pequeno ou grande porte, segundo ISACA \cite{Cobit}. 
    COBIT est� dividido em quatro dom�nios:
     
     \begin{itemize} 
     \item Planejamento e Organiza��o: trata as estrat�gicas e t�ticas que TI 
     possa contribuir para realiza��o dos objetivos de neg�cio; 
     \item Aquisi��o e Implementa��o: determina a estrat�gia de TI para
      identificar, qualificar e escolher solu��es a serem desenvolvidas ou adquiridas; 
      \item Entrega e Suporte: verificar os servi�os requeridos pelos processos
      de neg�cio, para que haja entrega de informa��es e suporte para as opera��o em 
      situa��es inesperadas; 
       \item Monitora��o: monitorar os controles da organiza��o de TI, sendo fundamental 
       avalia��o continua e regular da qualidade e da conformidade dos controles implantados.
       \end{itemize}

 	Segundo ISACA \cite{Cobit} COBIT �  baseado em 5 princ�pios que permitem que a 
 	organiza��o construa um framework de governan�a e gest�o de TI baseado em um 
 	conjunto de 7 habilitadores que otimizam investimentos em tecnologia e informa��o 
 	utilizados para o benef�cio dos interessados:
  \begin{itemize} 
    \item Princ�pio 1. Atender as necessidades dos stakeholders: As organiza��es
    existem para criando valor para os envolvidos.
    \item Princ�pio 2. Cobrir a organiza��o de ponta a ponta: os gestores de neg�cio 
    t�m a responsabilidade de tratara TI como um ativo estrat�gico, onde TI � igual aos 
    demais.
    \item Princ�pio 3. Aplicar um framework �nico e integrado: integrar
    todos os conhecimentos existentes em diferentes frameworks.
    \item Princ�pio 4. Possibilitar uma abordagem hol�stica: O COBIT utiliza de
    7 viabilizadores(Princ�pios, Pol�ticas e Frameworks, Processos, Estruturas 
    Organizacionais, Cultura, �tica e Comportamento, Informa��o, Servi�os, 
    Infraestrutura e Aplica��es, Pessoas, Habilidades e Compet�ncias) que apoiam a 
    governan�a e a gest�o de TI abordar a organiza��o de ma forma completa.
    \item Princ�pio 5. Separar a governan�a da gest�o: Determinar onde cada �rea atuara, suas 
    estruturas e prop�sitos.
    \end{itemize} 
 
    
    \subsection{ITIL ( \emph{Information Technology Infrastructure Library})}
   
    ITIL � a jun��o das melhores pr�ticas para auxiliar organiza��es tanto no
    setor publico como no privado, usando as melhores formas para a cria��o execu��o 
    e manuten��o de servi�os, dessa forma entregar valores aos clientes, assim facilitar 
    os resultados desejados. BALDIN \cite{Baldin}
    
    A ITIL teve seu inicio nos anos 80 pelo CCTE (Central Computer and Tele
    communications Agency), para atender necessidade do governo brit�nico onde a 
    insatisfa��o com a TI era preocupante. 
    Assim, foi desenvolvido um conjunto de boas pr�ticas para gerenciar a utiliza��o 
    eficiente e respons�vel dos recursos de TI, independendo dos fornecedores sendo 
    aplic�vel a qualquer organiza��o respeitando as necessidades especificas de cada uma 
    delas. ITIL � um framework que descreve as melhores pr�ticas no gerenciamento de 
    servi�o de TI. ITIL fornece para a governan�a de TI um framework para gerenciamento 
    e controle de TI, focando no uso de m�tricas e melhoria da qualidade dos servi�os, 
    podendo fornecer benef�cios para os diretores, aumentando a satisfa��o dos usu�rios 
    e clientes, melhorando a tomada de decis�o e diminuindo os riscos FILHO \cite{ITIL}.

 


%\end{referencialTeorico}
  \section{Governan�a de TI na UFSM campus de FW}
\label{sec:modeloPropost}

Governan�a de TI � uma forma de organizar as informa��es de uma organiza��o
sendo isso realizar o estudo no Campus de FRederico Westphalen vem a
complementar essa id�ia j� que � uma istitui��o de ensino p�blica e carrente de
tais recursos para gerir suas informa��es.

\subsection{ISO/IEC38500 e seus Princ�pios}
 Como fundamenta��o para o estudo de caso em quest�o usar-se-� a norma
 ISO/IEC38500 \cite{iso}, esta norma oferece princ�pios para orientar os
 dirigentes da organiza��o, sobre o uso eficaz, eficiente e aceit�vel da
 Tecnologia de Informa��o. Entre os princ�pios da norma ISO ser�o utilizados:

 \begin{itemize}
   \item Responsabilidade: Os indiv�duos e grupos dentro da organiza��o compreendem e 
 aceitam suas responsabilidades em rela��o a TI, tendo autonomia, concluindo que pessoas 
 s�o os principais ativos de qualquer organiza��o O princ�pio Responsabilidade da norma 
 ISO/IEC38500 fundamenta a vis�o das organiza��es e as exig�ncias que s�o feitas em 
 rela��o as pessoas que nela s�o inseridas \cite{iso}.
   \item Estrat�gia: levam em conta as capacidades atuais e futuras de TI. Os planos 
 estrat�gicos para TI satisfazem as necessidades atuais e cont�nuas da estrat�gia de 
 neg�cio da organiza��o. Envolvendo um posicionamento, pesquisa de produto, pra�a, pre�o e 
 poss�veis promo��es A responsabilidade � muito importante dentro da organiza��o, � atrav�s 
 dela que cada fun��o � exercida e executada, os diretores devem acompanhar o 
 desenvolvimento das atividades, \cite{iso}.
 \end{itemize}
   
 %\begin{itemize}
 
 A estrat�gia direciona para obter vantagens competitivas, tentando sempre 
alinhar TI com o neg�cio da organiza��o. Assim como a Responsabilidade a Estrat�gia
 tamb�m segue o modelo onde avaliar, direcionar e monitorar assume um papel importante, 
 diretores devem avaliar se o desenvolvimento da TI est� em conformidade com as reais 
 necessidades do neg�cio, podendo ser analisado na ferramenta CITSMART que
 responsabilidade � plenamente aplicada podendo ser direcionado para
 determinado grupo de trabalhos e dessa forma atribuir tais responsabilides,
 tamb�m � poss�vel delimitar estrat�gias de trabalho usando o framework da ITIL
 atrv�s da ferramenta CITISMART, dessa forma tornando as a��es controladas e
 sendo poss�vel retornar relat�rios para analise.

 \subsection{Mapeamento de Servi�os usando ITIL} 
 
   A ITIL~\cite{ITIL2} diz que o mapeamento de servi�o � estabelecer como o
servi�o vai proceder, como vai ser controlado e desenhado. Para isso � necess�rio 
uma esquematiza��o, utilizando Portf�lio de Servi�os e Catalogo de Servi�os.
 
   A Figura ~\ref{figura1} representa um modelo de Governan�a de TI, com seus
principais frameworks de governan�a, COBIT, ITIL e ISO/IEC38500, onde esses frameworks 
suportam ferramentas de implanta��o de Governan�a de TI, desenvolve processos de Governan�a 
e envolvem pessoas que comunicam se entre si para desenvolver processos e trabalhar com as 
ferramentas propostas, para atingir objetivo estabelecidos, tamb�m esta colocado na Figura
 ~\ref{figura1} um esbo�o da  Universidade Federal de Santa Maria que al�m do
 campus principal em Santa Maria ainda possui 4 campus distribu�dos no interior do 
 estado, dessa forma a figura demonstra que o estudo est� sendo realizado em um desses 
 campus do interior do estado, mais especificadamente o campus de Frederico Westphalen, 
 este estudo buscando identificar ferramentas j� utilizadas, poss�veis ferramentas que 
 poder�o ser suportadas pela mesma.
  
\begin{figure}[H]
\centering
\includegraphics[scale=0.5]{figuras/ModeloGTI.jpg}
\caption{Demonstrativo de estudo de Governan�a de TI em rela��o a UFSM
 campus de Frederico Westphalen }
\label{figura1}  
\end{figure}
 
   \subsubsection{Portf�lio de Servi�os}
   Portf�lio de Servi�os � um conjunto 
   completo de servi�os que ser�o entregues. S�o agrupados por tamanho, 
   disciplina e valor estrat�gico ou seja, o Portf�lio engloba todos os servi�os
   entregues pela organiza��o, ou pela �rea de TI da mesma, tamb�m os que est�o aposentados ou absoleto por de certa forma ainda ser �til para a organiza��o e por �ltimo elemento do portf�lio que n�o podemos esquecer, al�m de servi�os
   ativos e aposentados, s�o os servi�os propostos ou em desenvolvimento. S�o aqueles que 
   ser�o ou n�o servi�os ativos um dia.  A gest�o de portf�lio tem como objetivo gerenciar os servi�os durante todo o ciclo 
   de vida do mesmo. A gest�o do portf�lio � um processo de car�cter estrat�gico e deve 
ser conduzida por uma fun��o que tenha autonomia na organiza��o de TI: cargos de 
diretoria a executivos. Portf�lio de Servi�o � a representa��o
   de todos os servi�os de TI. O gerenciamento do portf�lio serve para organizar investimentos a 
   ser feitos na organiza��o. O Portf�lio de Servi�o est� dividido em tr�s
   partes \cite{ITIL2}:
   \begin{itemize}
   \item O funil de servi�o, ou pipeline de servi�o: que mostra oque est� por
   ser realizado; 
   \item O cat�logo de servi�o: mostra os servi�os que est�o em
    desenvolvimento;
   \item Os servi�os obsoletos: mostra os servi�os que devem ser descartados;
   \end{itemize}
      
 
     Para Filho ~\cite{ITIL} no Portf�lio de Servi�os devem estar posto todos os
     servi�os existentes no Cat�logo de Servi�os da organiza��o. O gerenciamento de servi�os inclui: definir, analisar, aprovar e controla.
     \begin{itemize}
     \item Definir: levantar os dados existentes no Portf�lio, verificando  oque melhorar para agregar valores;
     \item Analisar: analisar as demandas de servi�os para identificar oque agrega valor � organiza��o; 
     \item Aprovar: aprovar o portf�lio proposto autorizando recursos e servi�os futuros;
     \item Controlar: comunicar decis�es, alocar recursos para o inicio de
     atividades, renovar o portf�lio 
    \end{itemize}
  
  A Figura ~\ref{figura 2} apresenta um fluxo de atendimento onde cada chamada tem 
 um in�cio e um fim, onde o usu�rio realiza um chamado atrav�s de email, ou atrav�s 
 do telefone, o t�cnico avalia o chamado e entra em contato com o usu�rio, se o 
 problema puder ser resolvido remotamente o problema estar� resolvido caso n�o seja 
 poss�vel o t�cnico realiza atendimento local, onde resolve o problema, caso n�o 
 consiga mesmo assim ent�o � procurado alternativas poss�vel sendo assim o problema 
 se encerra. A ferramenta Help Desk utilizada pelos t�cnicos do campus de Frederico 
 Westphalen segue enfim esse processo. 
 
  %footnote{Figura etirada do endereço
 %\url{www.getinews.com.br}}
\begin{figure}[H]
\centering
\includegraphics[scale=0.4]{figuras/fluxoAtividade.jpg}
\caption{Fluxo de atividade de atendimento realizados pela equipe de TI} 
\label{figura 2}  
\end{figure}
 
  \subsubsection{Cat�logo de Servi�os} 
  ITIL \cite{ITIL2} define cat�logo de servi�os como "parte do Portf�lio dispon�vel
   para um cliente. S�o os servi�os ativos na vis�o do cliente em espec�fico, este 
   cliente ser� representado por uma organiza��o com a qual mant�m contrato. 
   Na gest�o do cat�logo, o objetivo � que todas as informa��es dos servi�os 
   ativos estejam claramente dispon�veis e especificadas para o clientes. O gestor 
   do processo tem um papel t�tico na presta��o dos servi�os de TI. 

O Cat�logo de Servi�os deve ser entendido como a principal ferramenta de comunica��o 
entre TI e Neg�cio, garantindo que os processos de demanda e oferta de servi�os 
sejam executados de forma eficaz. Um bom Cat�logo de Servi�os � o primeiro passo 
para uma boa Gest�o de Servi�os de TI.
  
  Ainda no ITIL \cite{ITIL2} um Cat�logo de Servi�os, al�m de importante 
  ferramenta de comunica��o e transpar�ncia entre TI e Neg�cio, deve ser 
  entendido como importante ferramenta gerencial na obten��o de informa��es 
  sobre a opera��o. O objetivo do Gerenciamento de Cat�logo de Servi�o � fornecer 
  uma �nica fonte de informa��es consistentes sobre todos os servi�os que est�o acordados para ser entregues aos clientes.
  % por um texto
  
  Ao realizar a analise da pesquisa realizada no setor de TI da universidade
  UFSM no campus de Frederico Westphalen com foco na ferramenta Help Desk
  utilizada pelos t�cnicos de TI, identificando assim, os tipos de servi�os
  prestados pelos membros do setor e quais eram suportados pela ferramenta, 
  dessa forma identificado que a ferramenta atende a apenas um tipo de servi�o 
  sendo ele a Abertura de Chamados, que no momento � usado para controle interno 
  do setor e que essa ferramenta n�o � institucionalizada, ou seja, a 
  popula��o acad�mica n�o faz uso da mesma.

 Nesta ferramenta s�o armazenados dados de diversas formas de atendimentos que s�o solicitados, entre eles instala��o de sistemas operacionais, manuten��o de programas 
instalados para os usu�rios, suporte de redes, em fim necessidades enfrentadas pelos 
usu�rios em rela��o a hardware e software em geral, identificando como cliente toda a 
comunidade acad�mica, professores, t�cnico, alunos e demais funcion�rios da universidade.

  A Tabela ~\ref{suportRedes}, apresenta as informa��es do atendimento 
  realizado na campus de Frederico Westphalen, servi�o este referente a Suporte 
  de Redes, nomeando o servi�o, definindo seu estado, o tipo de servi�o, quem s�o 
  os clientes que s�o beneficiados pelo servi�o, justificando a realiza��o do mesmo, 
  os resultados esperados se existe algum tipo de depend�ncia em rela��o a outros 
  servi�os, se existe algum tipo de mudan�as em projeto para tal servi�o e as
  necessidades de aperfei�oamento dos membros realizadores do servi�o.
 %Figura~\ref{figura 3} Portf�lio de Servi�os %Titulo da figura
  
%\begin{figure}[H]
%\centering
%\includegraphics[scale=0.7]{figuras/Portifolio.jpg}
%\caption{Portf�lio de Servi�os da Ferramenta Help Desk de Abertura de Chamados
%no Campus de Frederico Westphalen}
%\label{figura 3}  
%\end{figure}
\input{tabela/suportRedes} 

A Tabela~\ref{suportSoftf}, est� sendo realizado um atendimento para realizar
suporte a sotwares, sendo relatado o processo da mesma forma que quando atendido
um servi�o de suporte a redes, onde � relatado o nome de servi�os, status do
mesmo, seus clientes, justificativa da realiza��o os resultados desejados e
mudan�as previstas para o servi�o.

\begin{table}[H]
\tiny
	\centering
	\caption{Analise de Portif�lio: Suporte de Softwares}
	\label{suportSoftf}
	\begin{tabular}{|l|l|}
		\hline
		Nome do Servi�o&Suporte e instala��o de Software  \\
		\hline
		Status Atual do Servi�o&O servi�o � realizado pela equipe de TI da universidade, equipe esta composta de tr�s colaboradores efetivos e dois bolsistas. \\
		\hline
		Tipo de Servi�o&Suporte a software, orienta��o de  uso de  programas, por exemplo. Instala��o de softwares necess�rios, para atendimento de necessidades. \\
		\hline
		Clientes&Toda comunidade acad�mica: Professores, t�cnicos, dire��o, alunos. \\
		\hline
		Descri��o do Servi�o&Manuten��o de software,  instala��o de softwares, treinamento com usu�rio para utilizar os software requisitado. \\
		\hline
		Justificativa&Conseguir atender os clientes, para que  dificuldades sejam solucionadas, os dados encontrados geram feedback  tendo controle do realizado. \\
		\hline
		Resultados desejados em termos de utilidade&Satisfa��o do usu�rio, em rela��o a os softwares . \\
		\hline
		Depend�ncia&O atendimento sofre depend�ncia da telefonia j� que os chamados s�o feitos atrav�s do telefone. \\
		\hline
		Mudan�as Planejadas para o servi�o&A falta de m�o de obra n�o possibilita a realiza��o de melhorias. \\
		\hline
		Refer�ncias a planos pertinentes&Necessidade de aperfei�oamento, para efetuar melhorias. \\
		\hline
		Business Case&Riscos decorrentes a depend�ncia . \\
		\hline
	\end{tabular}
	\label{tab:Nome_Da_Tabela}
\end{table} 

A Tabela ~\ref{suportHard} identifica a realiza��o de um atendimento pra
suporte de hardware onde � nomeado o servi�o, identificado seu estatus, quem s�o seus
clientes, descrevendo suas caracteristicas, justificando a realiza��o de tal
atendimento, se ele possui alguma depend�ncia em rela��o a outros servi�os e se
h� previs�o de mudan�as para o servi�o.

\begin{table}[H]
\tiny
\centering
	\label{suportHard}
	\begin{tabular}{|l|l|}
		\hline
		Nome do Servi�o&Suporte a Hardware \\
		\hline
		Status Atual do Servi�o&O servi�o � realizado pela equipe de TI da universidade, equipe esta composta de tr�s colaboradores efetivos e dois bolsistas. \\
		\hline
		Tipo de Servi�o&Suporte a Hardware, atendimento a eventuais problemas que ocorram nas maquinas da universidade. \\
		\hline
		Clientes&Toda comunidade acad�mica: Professores, t�cnicos, dire��o, alunos. \\
		\hline
		Descri��o do Servi�o&O servi�o � realizado sempre que solicitado, onde � realizado manuten��o de hardware, onde s�o corrigidos eventuais problemas. \\
		\hline
		Justificativa&Atender os clientes, para que as dificuldades enfrentadas sejam solucionadas, os dados encontrados geram feedback  tendo controle do que foi realizado. \\
		\hline
		Resultados desejados em termos de utilidade&Satisfa��o do usu�rio, em rela��o a rede, para que assim possam realizar suas atividades  em perfeitas condi��es. \\
		\hline
		Depend�ncia&O atendimento sofre depend�ncia da telefonia j� que os chamados s�o feitos atrav�s do telefone. \\
		\hline
		Mudan�as Planejadas para o servi�o&A falta de m�o de obra n�o possibilita a realiza��o de melhorias.  \\
		\hline
		Refer�ncias a planos pertinentes&Necessidade de aperfei�oamento, para efetuar melhorias. \\
		\hline
		Business Case&Riscos decorrentes a depend�ncia. \\
		\hline
	\end{tabular}
	\label{tab:Nome_Da_Tabela}
	\caption{An�lise de Portif�lio: Suporte de Hardwares}
\end{table}

A Tabela~\ref{terceiro} Est� descrito uma das atividades desenvolvidas pelo 
setor de TI do campus de Frederico Westphalen onde o atendimento a terceiro est� 
relacionado com auxiliar as empresas terceirizadas que iram prestar servi�os na 
universidade prestando informa��es para que o objetivo da a��o seja atingida. 
Dessa norma nomeia o servi�o, estabelece o estado de tal servi�o, quem s�o os 
clientes desse servi�o, justificando a exist�ncia desse servi�o identificando 
qual o resultado esperado, suas depend�ncias.  

\begin{table}[h]
\tiny
	\centering
	
		\label{terceiro}
	\begin{tabular}{|l|l|}
		\hline
		Nome do Servi�o&Atendimento a Terceiros \\
		\hline
		Status Atual do Servi�o&O servi�o � realizado pela equipe de TI da universidade, equipe esta composta de tr�s colaboradores efetivos e dois bolsistas. \\
		\hline
		Tipo de Servi�o&Suporte o atendimento a terceiros, em eventuais problemas que ocorram nas atividades terceirizadas da universidade. \\
		\hline
		Clientes&Toda comunidade acad�mica: Professores, t�cnicos, dire��o, alunos. \\
		\hline
		Descri��o do Servi�o&O servi�o � realizado sempre que solicitado, onde � realizado presta��o de informa��es para que seja, corrigidos eventuais problemas. \\
		\hline
		Justificativa&Conseguir atender os clientes, para que dificuldades enfrentadas sejam solucionadas, gerando  feedback  para  \\
		\hline
		Resultados desejados em termos de utilidade&Satisfa��o do usu�rio, em rela��o a rede, para que assim possam realizar suas atividades  em perfeitas condi��es. \\
		\hline
		Depend�ncia&O atendimento sofre depend�ncia da telefonia j� que os chamados s�o feitos atrav�s do telefone. \\
		\hline
		Mudan�as Planejadas para o servi�o&A falta de m�o de obra n�o possibilita a realiza��o de melhorias.  \\
		\hline
		Refer�ncias a planos pertinentes&Necessidade de aperfei�oamento, para efetuar melhorias. \\
		\hline
		Business Case&Riscos decorrentes a depend�ncia. \\
		\hline
	\end{tabular}
	\label{tab:Nome_Da_Tabela}
	\caption{An�lise de Portif�lio: Suporte de Terceiros}
\end{table}

  


%A Tabela ~\ref{suportRedes} apresenta dados referentes a um per�odo de um ano
% apresentados bimestralmente, organizados em atividades resolvidas at� o prazo,
 %resolu��o do problema, satisfa��o dos usu�rios e atividades realizadas.
 
 
 \subsubsection{Servi�o de Help Desk}
 Segundo Statdlober \cite{J} o servi�o de Help Desk fornece um ponto 
 central de contato entre clientes e funcion�rios para apresentar os incidentes 
 e solicita��es de servi�os, permitindo a organiza��o oferecer um servi�o de alta 
 qualidade a um custo operacional m�nimo. Atrav�s de relat�rios centralizados e da 
 monitora��o de solicita��es automatizadas � poss�vel gerar valor ao neg�cio aumentando 
 consideravelmente a produtividade e reduzindo custos.
 % aqui a Figura desen
 
A Figura~\ref{figura4} apresenta a interface do sistema Help Desk usado pelo 
setor de TI da universidade UFSM campus de Frederico Westphalen, nesta interface 
e solicitado os dados do usu�rio que realizar� a chamada e tamb�m os dados do local
 onde o problema est� instaurado e os dados do equipamento que est� com problema.
  Nos dados de identifica��o � solicitado que o usu�rio coloque seu email, nome e
 matr�cula, para fins de controle, tamb�m nos dados de localiza��o � solicitado uma 
 descri��o do local onde est� instaurado o problema para dessa forma, facilitar o 
 atendimento caso haja a necessidade de deslocamento ao local e tamb�m poder ser 
 identificado posteriormente num feedback os lugares de maior incid�ncia de problemas, 
 posteriormente a descri��o identificada pelo usu�rio do que est� acontecendo com o 
 equipamento, assim o t�cnico possa tomar atitudes para resolver o problema. 
 %Figura~\ref{figura4} Interface Sistema Help Desk%Titulo da figura
  
\begin{figure}[H]
\centering
\includegraphics[scale=0.5]{figuras/desen.jpg}
\caption{Interface do Sistema Help Desk usado na UFSM
campus Frederico Westphalen}
\label{figura4}  
\end{figure}

A ferramenta Help Desk n�o � institucionalizada por ser usada somente para
controle interno do setor de TI, dessa forma os dados n�o s�o totalmente atualizados 
e podem n�o conter todas as informa��es existentes.
 
 \subsubsection{Gest�o de Incid�ncias }
 
 No ITIL \cite{ITIL2} o Gerenciamento de Incidentes tem como foco principal reestabelecer servi�o, minimizando o impacto negativo no neg�cio, uma solu��o de contorno ou reparo r�pido fazendo com que o cliente volte a trabalhar. Garantir que os melhores n�veis de disponibilidade e de qualidade dos servi�os, mantendo os acordos de n�vel de servi�o � tamb�m uma tarefa da ger�ncia de incidentes.
 % Aqui a figura Tabela1.pdf
 
 A Figura~\ref{figura5} apresenta as  atividades do Sistema de Chamadas com as
 atividades  realizadas pelo setor de TI da universidade e cadastradas no Sistema, 
 dados este coletados no decorrer de um ano, onde est�o organizados bimestralmente, 
 ou seja os relat�rios s�o apresentados de dois em dois meses, descritos em n�meros  
 decimais e acompanhados do porcentual que cada atividade representa em um �mbito geral.

Ao analisar a figura � poss�vel detectar que o principal motivo para abertura de
chamadas � relacionadas as impressoras da universidade, em seguida est�o problemas
relacionados � rede da universidade, tamb�m � poss�vel observar que existem servi�os que nunca tiveram 
solicita��o de chamados. Tamb�m � poss�vel observar que, nos anos de janeiro e
fevereiro, correspondente aos meses de f�rias letivas as chamadas s�o menores,
devido a menor utiliza��o de equipamentos 
 
 %Figura~\ref{figura5} Sistema de Chamadas dos meses  %Titulo da figura 
%  % Please add the following required packages to your document preamble:
% \usepackage[normalem]{ulem}
% \useunder{\uline}{\ul}{}
\begin{table}[h]
\tiny

\centering
\caption{Sistema de Chamada dos Meses}
\label{sistChamada}
\begin{tabular}{|l|l|l|l|l|l|l|l|l|l|l|l|l|}
\hline
\multicolumn{13}{|c|}{Sistema de Chamada dos meses}                                                                                                                                                                                                                                            \\ \hline
\multicolumn{13}{|l|}{}                                                                                                                                                                                                                                                                        \\ \hline
M�s/Ano                                                              & \multicolumn{2}{l|}{Maio/Jun 2014} & \multicolumn{2}{l|}{Jul /Ago 2014} & \multicolumn{2}{l|}{Set/Out 2014} & \multicolumn{2}{l|}{Nov/Dez 2014} & \multicolumn{2}{l|}{Jan/Fev 2015} & \multicolumn{2}{l|}{Mar/Abr 2015} \\ \hline
\multicolumn{13}{|l|}{}                                                                                                                                                                                                                                                                        \\ \hline
Atividades Atuais                                                    & NUM           & \%                 & NUM           & \%                 & NUM           & \%                & NUM           & \%                & NUM           & \%                & NUM           & \%                \\ \hline
Instala��o/ problemas desw b�sicos (SO,Navegador, Antiv�rus)         & 6             & 17,14\%            & 6             & 13,33\%            & 2             & 11,11\%           & 1             & 5,56\%            & 3             & 60,00\%           & 1             & 7,69\%            \\ \hline
Drivers/Codecs                                                       & 2             & 5,71\%             & 2             & 44,44\%            & 0             & 0,00\%            & 0             & 0,00\%            & 1             & 20,00\%           & 0             & 0,00\%            \\ \hline
SIE                                                                  & 2             & 5,71\%             & 5             & 11,11\%            & 1             & 5,56\%            & 1             & 5,56\%            & 1             & 20,00\%           & 2             & 15,38\%           \\ \hline
Ponto de rede                                                        & 1             & 2,86\%             & 1             & 2,22\%             & 0             & 0,00\%            & 0             & 0,00\%            & 0             & 0,00\%            & 0             & 0,00\%            \\ \hline
Impressora                                                           & 5             & 14,29\%            & 5             & 11,11              & 2             & 11,11\%           & 7             & 38,89\%           & 0             & 0,00\%            & 5             & 38,46             \\ \hline
Formata��o                                                           & 3             & 8,57\%             & 0             & 0,00\%             & 0             & 0,00\%            & 0             & 0,00\%            & 0             & 0,00\%            & 0             & 0,00\%            \\ \hline
Ativa��o do proxy                                                    & 0             & 0,00\%             & 0             & 0,00\%             & 1             & 5,56\%            & 0             & 0,00\%            & 0             & 0,00\%            & 0             & 0,00\%            \\ \hline
Troca de Perif�rico                                                  & 2             & 5,71\%             & 2             & 4,44\%             & 2             & 11,11\%           & 2             & 11,11\%           & 0             & 0,00\%            & 0             & 0,00\%            \\ \hline
Instala��o/remo��o de softwares espec�ficos (statistic, sigepweb,..) & 0             & 0,00\%             & 1             & 2,22\%             & 3             & 16,67\%           & 0             & 0,00\%            & 0             & 0,00\%            & 0             & 0,00\%            \\ \hline
Cadastrar MAC                                                        & 0             & 0,00\%             & 0             & 0,00\%             & 0             & 0,00\%            & 0             & 0,00\%            & 0             & 0,00\%            & 0             & 0,00\%            \\ \hline
Softwares de escrit�rio(Office, Leitor de PDF)                       & 0             & 0,00\%             & 3             & 6,67\%             & 1             & 5,56\%            & 4             & 22,22\%           & 0             & 0,00\%            & 0             & 0,00\%            \\ \hline
Sistemas do governo(SIAFI, SCDP, Token)                              & 0             & 0,00\%             & 0             & 0,00\%             & 0             & 0,00\%            & 0             & 0,00\%            & 0             & 0,00\%            & 0             & 0,00\%            \\ \hline
Remo��o v�rus/ Spyware                                               & 2             & 5,71\%             & 2             & 4.44\%             & 2             & 11,11\%           & 1             & 5,56\%            & 0             & 0,00\%            & 1             & 7,69\%            \\ \hline
Instala��o decomputadores                                            & 4             & 11,43\%            & 4             & 8,89               & 0             & 0,00\%            & 0             & 0,00\%            & 0             & 0,00\%            & 1             & 7,69              \\ \hline
Instala��o/troca oumanuten��o de roteador                            & 0             & 0,00\%             & 0             & 0,00\%             & 0             & 0,00\%            & 0             & 0,00\%            & 0             & 0,00\%            & 0             & 0,00\%            \\ \hline
Rede                                                                 & 5             & 14,29\%            & 6             & 13,33\%            & 1             & 5,56\%            & 0             & 0,00\%            & 0             & 0,00\%            & 1             & 7,69\%            \\ \hline
Ponto Eletr�nico                                                     & 0             & 0,00\%             & 0             & 0,00\%             & 0             & 0,00\%            & 0             & 0,00\%            & 0             & 0,00\%            & 0             & 0,00\%            \\ \hline
Conflito de IP                                                       & 0             & 0,00\%             & 0             & 0,00\%             & 0             & 0,00\%            & 0             & 0,00\%            & 0             & 0,00\%            & 0             & 0,00\%            \\ \hline
Datashow / Projetores                                                & 0             & 0,00\%             & 2             & 4,44\%             & 0             & 0,00\%            & 0             & 0,00\%            & 0             & 0,00\%            & 0             & 0,00\%            \\ \hline
Configura��o de e-mail                                               & 0             & 0,00\%             & 0             & 0,00\%             & 0             & 0,00\%            & 0             & 0,00\%            & 0             & 0,00\%            & 0             & 0,00\%            \\ \hline
Outros                                                               & 3             & 8,57\%             & 6             & 13,33\%            & 3             & 16,67\%           & 2             & 11,11\%           & 0             & 0,00\%            & 2             & 15,38\%           \\ \hline
Total                                                                & 35            & 100,00\%           & 45            & 100,00\%           & 18            & 100,00\%          & 18            & 100,00\%          & 5             & 100,00\%          & 13            & 100,00\%          \\ \hline
\end{tabular}
\end{table}
  
 
\begin{figure}[H]
\centering
\includegraphics[scale=0.7]{figuras/tabela1.jpg}
\caption{A figura demonstra as atividades realizadas no periodo de um ano}
\label{figura5}  
\end{figure}



 % aqui a figura Tabela
 
 %Figura~\ref{figura6} Indicadores do Sistema de Chamadas %Titulo da figura
  
%\begin{figure}[H]
%\centering
%\includegraphics[scale=0.5]{figuras/Tabela.jpg}
%\caption {Demonstrativo do Sistema de Chamadas}
%\label{figura6}  
%\end{figure}
\begin{table}[H]
\tiny
\centering
\begin{tabular}{|l|l|l|l|l|l|l|l|}
\hline
\multicolumn{8}{|l|}{Indicadores do Sistema de Chamadas do Campus de FW da UFSM}                                                                        \\ \hline
                    & \multicolumn{1}{c|}{At. Prazo} & \multicolumn{2}{c|}{Resolu��o} & \multicolumn{2}{c|}{Satisfa��o} & \multicolumn{2}{c|}{Avaliadas} \\ \hline
M�s (Dados Atuais)  & \%                             & \%            & Co             & \%            & Av.             & \%            & Convite        \\ \hline
Maio e Junho 2014   & 76,12\%                        & 100\%         & 67 de 67       & 0,00\%        & 0 de 0          & 0,00\%        & 0 de 0         \\ \hline
Julho e Agosto 2014 & 78,15\%                        & 99,16\%       & 118 de 119     & 100\%         & 46 de 46        & 46,00\%       & 46 de 100      \\ \hline
Set e Out 2014      & 67,39\%                        & 98,91\%       & 91 de 92       & 100\%         & 35 de 35        & 43,75\%       & 35 de 80       \\ \hline
Nov e Dez 2014      & 74,51\%                        & 98,04\%       & 50 de 51       & 100\%         & 22 de 22        & 45,83\%       & 22 de 48       \\ \hline
Jan e Fev 2015      & 73,33\%                        & 93,33\%       & 14 de15        & 100\%         & 4 de 4          & 26,67\%       & 4 de 15        \\ \hline
Mar�o e Abril 2015  & 75\%                           & 100,00\%      & 36 de 36       & 100\%         & 15 de 15        & 44,12         & 15 de 34       \\ \hline
\end{tabular}
\caption{Indicadores do Sistema de Chamadas}
\label{tabindicador1} 
\end{table}



A Tabela ~\ref{indicador} apresenta dados referentes a um per�odo de um ano apresentados bimestralmente, organizados em atividades resolvidas at� o prazo, resolu��o do problema, satisfa��o dos usu�rios e atividades realizadas. 
 
 Nas atividades realizadas at� o prazo o percentual est� sendo colocado desconsiderando 
 que as atividades s�o armazenadas em um prazo de 48 horas corridos n�o considerando assim 
 s�bados e domingos, nem feriados dessa forma o percentual n�o � preciso. 

� poss�vel observar que a resolu��o dos problemas ocorridos � muito satisfat�ria j� que em quase todos os meses e quase atingido totalmente.  A satisfa��o em rela��o aos atendimentos � total em todos os meses, e o percentual de pessoas que avaliam os atendimentos s�o vari�veis, j� que o �ndice de avalia��o e baixo em propor��o a quantidade de solicita��es de atendimento. Tamb�m � poss�vel observar que em meses que correspondem a f�rias escolares o atendimento � reduzido consideravelmente, devido a um menor uso de hardware e software no campus.

  %\begin{table}[H]
\tiny
\centering
\begin{tabular}{|l|l|l|l|l|l|l|l|}
\hline
\multicolumn{8}{|l|}{Indicadores do Sistema de Chamadas do Campus de FW da UFSM}                                                                        \\ \hline
                    & \multicolumn{1}{c|}{At. Prazo} & \multicolumn{2}{c|}{Resolu��o} & \multicolumn{2}{c|}{Satisfa��o} & \multicolumn{2}{c|}{Avaliadas} \\ \hline
M�s (Dados Atuais)  & \%                             & \%            & Co             & \%            & Av.             & \%            & Convite        \\ \hline
Maio e Junho 2014   & 76,12\%                        & 100\%         & 67 de 67       & 0,00\%        & 0 de 0          & 0,00\%        & 0 de 0         \\ \hline
Julho e Agosto 2014 & 78,15\%                        & 99,16\%       & 118 de 119     & 100\%         & 46 de 46        & 46,00\%       & 46 de 100      \\ \hline
Set e Out 2014      & 67,39\%                        & 98,91\%       & 91 de 92       & 100\%         & 35 de 35        & 43,75\%       & 35 de 80       \\ \hline
Nov e Dez 2014      & 74,51\%                        & 98,04\%       & 50 de 51       & 100\%         & 22 de 22        & 45,83\%       & 22 de 48       \\ \hline
Jan e Fev 2015      & 73,33\%                        & 93,33\%       & 14 de15        & 100\%         & 4 de 4          & 26,67\%       & 4 de 15        \\ \hline
Mar�o e Abril 2015  & 75\%                           & 100,00\%      & 36 de 36       & 100\%         & 15 de 15        & 44,12         & 15 de 34       \\ \hline
\end{tabular}
\caption{Indicadores do Sistema de Chamadas}
\label{tabindicador1} 
\end{table}

  
%\ldots
 
  \section{Roteiro de Implanta��o}
\label{sec:roteiroDeImplantacao} 
 
 Planejamento � a palavra chave para as organiza��es, pois atrav�s dele h� a possibilidade de 
 abrang�ncias competitivas, quando n�o existe planejamento estrat�gico n�o h� conscientiza��o 
 da dire��o em rela��o ao plano estrat�gico de TI e sua necessidade para sustentar metas de 
 neg�cio. O planejamento possibilita a aquisi��o de maturidade \cite{Cobit}.

Para que seja poss�vel acontecer a Governan�a de TI � necess�ria uma sensibiliza��o de 
todas as partes envolvida na organiza��o, mas principalmente da dire��o, pois � atrav�s 
dela que eventuais problemas poder�o ser solucionados ou amenizados, \cite{f}.
 
 Com base no modelo proposto por \cite{f} torna se poss�vel desenvolver um
 roteiro  � ser proposto para desenvolver na UFSM campus de Frederico
 Westphalen onde foi selecionado cinco passos de implanta��o e.
 
  \subsection{Mapeamento de Servi�os usando ITIL} 
 
   A ITIL~\cite{ITIL2} diz que o mapeamento de servi�o � estabelecer como o
servi�o vai proceder, como vai ser controlado e desenhado. Para isso � necess�rio 
uma esquematiza��o, utilizando Portf�lio de Servi�os e Catalogo de Servi�os.
 
   A Figura ~\ref{figura1} representa um modelo de Governan�a de TI, com seus
principais frameworks de governan�a, COBIT, ITIL e ISO/IEC38500. A figura
 ~\ref{figura1} apresenta um esbo�o da  UFSM que
al�m do campus principal em Santa Maria ainda possui 4 campus distribu�dos no interior do 
 estado, dessa forma a figura demonstra que o estudo est� sendo realizado em um desses 
 campus do interior do estado, mais especificadamente o campus de Frederico Westphalen, 
 este estudo buscando identificar ferramentas j� utilizadas, poss�veis ferramentas que 
 poder�o ser suportadas pela mesma.
  
\begin{figure}[H]
\centering
\includegraphics[scale=0.5]{figuras/ModeloGTI.jpg}
\caption{Demonstrativo de estudo de Governan�a de TI em rela��o a UFSM
 campus de Frederico Westphalen }
\label{figura1}  
\end{figure} 
 
 A Figura~\ref{f19} apresentado um poss�vel plano de implanta��o de
Governan�a de TI no Campus de Frederico Westphalen delimitando ac�es que
direcionariam a implanta��o da mesma, dessa forma seriam oito passo que
trilhariam as a��es at� chegar a aprova��o do programa, em en sequencia a
elabora��o de planos de gerencia de mudan�as e aprimoramento da implanta��o,
implantando os projetos, monitorando os mesmos, avaliando para dessa forma
comunicar os resultados.
 %\begin{itemize}
\begin{figure}[H] 
\centering
\includegraphics[scale=0.6]{figuras/rotImpl.png}
\caption{Projeto de Roteiro de Implanta��o no Campus de FW }
\label{f19}  
\end{figure}

\begin{itemize}
\item Sensibilizar a dire��o: ter o apoio da dire��o para dessa forma conseguir
angariar recursos para desenvolve as atividades necess�rias na implanta��o de Governan�a 
de TI. Para fundamentar essa a��o pode ser usado de
v�rios instrumentos entre eles palestras, visitas a outras organiza��es que j� fazem
uso de Governan�a de TI, mostrando poss�veis resultados que a mesma pode gerar;
\item Definir pap�is e definir responsabilidades: definir quem realizara cada. 
	atividade e a responsabilidade que ter� em rela��o a tal atividade para dessa 
	forma atingir ao objetivo final, (alta administra��o, �rea de auditoria,
compliance e riscos, desenvolvimento, suporte, infraestrutura tecnol�gica, seguran�a
da informa��o);
\item Estabelecer metas e indicador final: Tra�aras a��es, determinar oque se quer
ter ao final dessas a��es e de que forma se pretende chegar ao final dessas a��es; 

\item Estabelecer um roadmap de implanta��o: estabelecendo a��es de curto m�dio
e longo prazo, identificando as a��es de maior prioridade. Identificar o que
precisa ser feito, definir a sequencia das atividades a ser implantadas, 
identificando os benef�cios que poder�o ser alcan�ados;
\item Elaborar o plano do programa: definir quais projetos faram parte do
programa, definir escopo. Ap�s ter o roadmap delimitar
quais projetos que faram parte do programa, delimitando escopo, estrutura
e a��es de TI, sequencia de implanta��o dos processos, em um linha de tempo pr�
estabelecida.
\item Aprovar o programa: Ap�s ter o plano pronto realizar a aprova��o do mesmo
para dar in�cio a implanta��o de Governan�a de TI; 
\item Implantar as a��es de mudan�a: por em pr�tica oque foi planejado;
\item Executar os projetos de implanta��o de Governan�a de TI: Executar oque
anteriormente foi planejado de forma organizada; 
\item Monitorar a implanta��o: Realizar o monitoramento para prever necessidades
d mudan�as ou realizar altera��es no plano;
\item Avaliar os resultados: Avaliar para definir os pr�ximos passos;
\item Comunicar os resultados alcan�ados: Tomar decis�es em conjunto � a melhor
 forma de se ter sucesso; 
 \item	Revisar e evoluir o programa: Adaptar o plano para dessa forma ter
 sucesso nos passos futuros.
\end{itemize}
%\end{referencialTeorico}

  \section{CITSMART}
\label{sec:citsmart}

 O Citsmart � um software de gest�o de TI, que implementa conceitos e
 tecnicas de Governan�a de TI.  Citsmart tem como objetivo manter a efici�ncia
 nos processos de presta��o de servi�os e promover sua melhoria \cite{Citsmart}.
 
A Figura~\ref{f15} demonstra a cria��o de servi�os no
software CITSMART esta figura est� demonstrada a cria��o de um servi�o de 
suporte de redes realizado na UFSM campus de FW servi�o este 
que est� sendo listado nas tabelas da se��o 5.4 , onde  foi obtido dados pela
pesquisa utilisando cat�logo de servi�os do ITIL dos sevi�os realizados pelo
setor de TI, demonstrando a possibilidade de utiliza��o desta ferramenta para
controle destes dados .

  \begin{figure}[H]
\centering
\includegraphics[scale=0.3]{figuras/cadastroRede.jpg}
\caption{Interface demonstrando o cadastro de servi�os  de Redes
CITSMART }
\label{f15}  
\end{figure} 

  A Figura~\ref{figura16} representa um exemplo ficticio de listagem de servi�os
  realizados e previmente cadastrados na ferramenta CITSMART. Dessa forma
  possibilitando uma melhor navega��o e visualiza��o da ferramenta.
  
\begin{figure}[H]
\centering
\includegraphics[scale=0.3]{figuras/listaServicos.jpg}
\caption{Interface de cadastro de servi�os  de Software CITSMART}
\label{figura16}  
\end{figure}
 A Figura ~\ref{figura8}  est� representada uma interface  do gerenciador de
cadastro de solicita��es e chamado de atendimento do sistema CITSMART, nessa interface 
est�o organizados os chamados com a numera��o do chamado o tipo de servi�o que foi solicitado, 
tamb�m se � incidente ou requisi��o e a data de cria��o de tal chamado, a prioridade do 
chamado e o prazo limite para a execu��o do chamado e tamb�m o estado que ele se encontra 
se em execu��o, encerado ou em prazo normal de execu��o. 

 \begin{figure}[H]
\centering
\includegraphics[scale=0.5]{figuras/solicitacao.jpg}
\caption{Interface de Gerenciador de Cadastro de Solicita��es }
\label{figura8}  
\end{figure}

A Figura ~\ref{figura9} apresenta um exemplo de relat�rios do sistema
CITSMART. Neste exemplo esta sendo posto os prazos de atendimento das atividades
cadastradas na ferramenta CITSMART, podendo estar eles em prazo normal, vencidos ou prazo 
a vencer, relacionando a quantidade de cada atividade a seu prazo. Tamb�m � colocada a 
prioridade de cada atividade que pode ser de 1 a 5 dependendo de sua import�ncia e dessa 
forma determinar qual devera ser realizado primeiro. Outra forma de relat�rio � gerada em 
gr�fico para uma melhor visualiza��o destas informa��es.
 \begin{figure}[H]
\centering
\includegraphics[scale=0.4]{figuras/solicitaPrio.jpg}
\caption{Interface de tipos de relat�rios gerados  pelo
software CITSMART}
\label{figura9}  
\end{figure}


 
  \section{An�lise do Question�rio Aplicado}
\label{sec:analise}

Nesta se��o s�o apresentados os resultados obtidos por meio da aplica��o de um
instrumento, que teve como objetivo colher informa��es e opini�es da pessoas que 
est�o envolvidas no cotidiano da UFSM campus de Frederico Westphalen  visando 
identificar a vis�o dos entrevistados sobre a infraestrutura tecnol�gica do mesmo, 
o conhecimento que tem sobre a import�ncia de se ter a equipe de TI em perfeitas 
condi��es de trabalho e o grau de satisfa��o que a comunidade acad�mica apresenta 
em rela��o a equipe e equipamentos dispon�veis no campus. O instrumento utilizado 
foi baseado no plano diretor de tecnologia da informa��o.

O instrumento proposto foi subdividido entre os temas Sistema de Informa��o, Servi�os 
Dispon�veis, Infraestrutura, Equipe de TI, Conhecimento que a comunidade acad�mica tem 
em rela��o a tais contextos, tamb�m o grau de satisfa��o da mesma em rela��o a estes 
processos. O mesmo foi aplicado em outubro do ano de 2015, pelo autor deste artigo, atrav�s 
do envio de um question�rio montado na ferramenta Google Forms, para toda a comunidade 
acad�mica, professores, t�cnicos, alunos. 

Num primeiro momento foi realizado um pergunta extra pesquisa do conte�do de
Governan�a de TI para saber se quem estava respondendo o question�rio tinha que papel 
na universidade. Em rela��o a esta pergunta onde foram obtido 59 respostas sendo
dessas 21 de alunos (35,6\%), 15 t�cnicos (25,4\%) e 23 professores (39\%), mostrando que
abrangeu todas as �reas da universidade.

A primeira quest�o. \textit{'' Em se tratando de Sistemas de Informa��o. Os
recursos de tecnologia de informa��o dispon�veis s�o suficientes para atender os objetivos da UFSM campus 
Frederico Westphalen''}. Foi dividido em t�picos.
\begin{itemize}
\item \textit{''Portais SIE-Web''} (Portal do Professor, do RH, Produ��o
Institucional, Portal do Aluno, etc.), obteve 59 respostas onde 4 pessoas responderam concordar 
totalmente com a quest�o (6,8\%), 42 concordaram (71,2\%), 2 manifestaram se
indiferentes (3,4\%) e 11 discordaram (18,6\%).
\item \textit{''P�gina Web da UFSM''}, obteve 59 respostas onde 9 pessoas
responderam concordar totalmente com a quest�o (15,3\%), 28 concordaram
(47,5\%), 8 manifestaram se indiferentes (13,6\%) e 14 discordaram (23,7\%).
\item \textit{'' Ferramentas de comunica��o (p�ginas web, blogs, not�cias, redes
sociais, etc.)''}, obteve 59 respostas onde 8 pessoas responderam concordar
totalmente com a quest�o (13,6\%), 31 concordaram (52,5\%), 8 manifestaram se indiferentes (13,6\%), 10
  discordaram (16,9\%) e 2 discordam totalmente (3,4\%).
\item \textit{'' Ferramentas de apoio � educa��o (apres. de slides, v�deos,
elabora��o de texto, etc.)''}, obteve 59 respostas onde 7 pessoas responderam
concordar totalmente com a quest�o (12,1\%), 26 concordaram (44,8\%), 10 manifestaram se
 indiferentes (17,2\%), 12 discordaram (20,7\%) e 3 discordam totalmente
 (5,2\%).
\item \textit{'' Moodle''}, obteve 59 respostas onde 10 pessoas responderam
concordar totalmente com a quest�o (16,9\%), 21 concordaram (35,6\%), 19 manifestaram
 se indiferentes (32,2\%), 7 discordaram (11,9\%) e 2 discordam
 totalmente(3,4\%).
\item \textit{''Outros aplicativos e/ou planilhas para controle interno''},
obteve 59 respostas onde 1 pessoas responderam concordar totalmente com a
quest�o (1,7\%), 17 concordaram (28,8\%), 30 manifestaram se indiferentes
(50,8\%), 10 discordaram (16,9\%) e 1 discordam totalmente (1,7\%). Neste mesmo
t�pica foram dadas varias sugest�es de poss�veis melhorias entre elas algum aplicativo
 que auxilie a fazer agendamentos no RU e um sistema para inscri��o em eventos 
 (que pudesse ser parametrizado para atender diferentes eventos da UFSM), al�m de um 
 sistema que permitisse realizar uma das partes da avalia��o institucional, que � a 
 avalia��o interna de cada curso junto aos discentes.
 \end{itemize}
   
\begin{figure}[H]
\centering
\includegraphics[scale=0.7]{figuras/questao11.jpg}
\caption{Gr�fico relacionado a quest�o 1 do question�rio de Governan�a de TI }
\label{figura10}  
\end{figure}
A Figura ~\ref{figura10} est� resumindo os dados relacionados a sistemas de
informa��o de TI na UFSM campus de Frederico Westphalen.
 
A segunda quest�o. \textit{''EM SE TRATANDO DE SERVI�OS. Os recursos de
Tecnologia de Informa��o dispon�veis s�o suficientes para atender os objetivos da UFSM
campus Frederico Westphalen''}.
 Foi dividido em t�picos.
 \begin{itemize}
\item \textit{''Suporte''}, obteve 59 respostas onde 8 pessoas responderam
concordar totalmente com a quest�o (13,6\%), 34 concordaram (57,6\%), 8
manifestaram se  indiferentes (13,6\%), 7 discordaram (11,9\%) e 2 discordam
totalmente (3,4\%).
\item \textit{''Capacita��o''}, obteve 59 respostas onde 6 pessoas responderam
concordar totalmente com a quest�o (10,2\%), 26 concordaram (44,1\%), 12 manifestaram se
indiferentes (20,3\%), 14 discordaram (23,7\%) e 1 discordam totalmente (1,7\%).
\item \textit{'' D�vidas, orienta��es, etc.''}, obteve 58 respostas onde 7
pessoas responderam concordar totalmente com a quest�o (12,1\%), 31 concordaram
(53,4\%), 9 manifestaram se indiferentes (15,5\%) e 11 discordaram (19\%).
Esta quest�o levanta a necessidade de melhorias, para assim poder se atingir de 
uma forma mais ampla todo o setor requente de servi�os.
\end{itemize}

\begin{figure}[H]
\centering
\includegraphics[scale=0.7]{figuras/questao22.jpg}
\caption{Gr�fico relacionado a quest�o 1 do question�rio de Governan�a de TI }
\label{figura11}  
\end{figure}
A Figura ~\ref{figura11} est� resumindo os dados relacionados a servi�os
de TI na UFSM campus de Frederico Westphalen.

A terceira quest�o.\textit{ ''EM SE TRATANDO DE INFRAESTRUTURA. Os recursos de
Tecnologia de Informa��o dispon�veis s�o suficientes para atender os objetivos da UFSM campus Frederico
 Westphalen''}. Foi dividido em t�picos.
 
 \begin{itemize}
\item \textit{''Equipamentos''}, obteve 59 respostas onde 2 pessoasresponderam
concordar totalmente com a quest�o (3,4\%), 21 concordaram (35,4\%), 2 manifestaram se indiferentes
(3,4\%), 27 discordaram (45,8\%) e 7 discordam totalmente (11,9\%).
\item \textit{''Redes''}, obteve 59 respostas onde 12 concordaram (20,3\%),
12 manifestaram se indiferentes (20,3\%), 25 discordaram (42,4\%) e 10 discordam
totalmente (16,9\%).
\item\textit{'' Internet''}, obteve 59 respostas onde 10 concordaram
(16,9\%), 5 manifestaram se indiferentes (8,5\%), 26 discordaram (44,1\%) e 18
discordam totalmente (30,5\%).
Essa quest�o mostra que a comunidade acad�mica est� � espera de melhorias, 
ou seja, h� a necessidade de investimento na melhoria tanto de redes, internet e 
equipamentos.
\end{itemize}

\begin{figure}[H]
\centering
\includegraphics[scale=0.7]{figuras/questao33.jpg}
\caption{Gr�fico relacionado a quest�o 1 do question�rio de Governan�a de TI }
\label{figura12}  
\end{figure}
A Figura ~\ref{figura12} est� resumindo os dados relacionados a infraestrutura
de TI na UFSM campus de Frederico Westphalen.

Na quest�o.\textit{'' Quanto � equipe de Tecnologia de Informa��o, Comentar
sobre a import�ncia:''}
foi  abordada a atua��o  da equipe e sua import�ncia para a UFSM campos de Frederico 
Westphalen, onde foi obtido como resposta. A equipe � de fundamental import�ncia, pois 
sem eles n�o tem como a Universidade caminhar, por exemplo, quando da problema na 
internet eles que tentam resolver ou entram em contato para resolverem, qualquer 
problema nos equipamentos s�o eles que s�o chamados, ou seja a universidade n�o anda 
sem essa equipe. A equipe do N�cleo de Inform�tica auxilia em muito nas atividades 
desempenhadas pelo nosso Departamento (DTecInf) e Curso (Sistemas de Informa��o), 
ajudando a manter os laborat�rios de inform�tica em funcionamento (hardware e software), 
al�m do apoio aos eventos, tais como a JASI (Jornada Acad�mica de Sistemas de Informa��o)
e o Encontro do GDG MAU (Grupo de Desenvolvedores Google do M�dio Alto Uruguai), entre 
outros.

Na quest�o.\textit{'' Coment�rios e sugest�es sobre o assunto: onde foi obtida
resposta como''}:
Melhorias nos sistemas de internet, telefonia, treinamentos para que professores de 
distintas �reas possam aplicar TI nas pr�ticas de ensino e pesquisa. Urgentemente 
providenciar espa�o para aulas � dist�ncia (considerando o isolamento f�sico do centro 
em rela��o � sede e outro campus). Melhorias de equipamentos para incentivar a fixa��o 
de professores em campus isolado.

Na pr�xima pergunta. \textit{'' Existem atividades importantes que a UFSM campus 
Frederico Westphalen est� deixando de realizar devido � falta ou precariedade dos recursos de 
Tecnologia de Informa��o dispon�veis''} obteve 57 respostas onde 22
entrevistados (38,6\%) disseram sim a universidade est� deixando de realizar atividades importantes, 
7(12,3\%) n�o e 28 (49,1\%) afirmaram n�o ter conhecimento sobre o assunto. Em
seguida foi questionado quais seriam essas atividades obtendo como resposta.  
Melhorias no ensino, facilidade de comunica��o, rela��es institucionais e extra 
institucionais, videoconfer�ncias, transmiss�es, eventos, agilidade nos processos e 
melhorias no acesso a internet.
\begin{figure}[H]
\centering
\includegraphics[scale=0.7]{figuras/questao6.jpg}
\caption{Gr�fico relacionado a quest�o 1 do question�rio de Governan�a de TI }
\label{figura13}  
\end{figure}
A Figura ~\ref{figura13} ilustra resumidamente a opini�o da cominidade
acad�mica em rela��o aos recursos de TI existentes na UFSM.

Na sequencia a quest�o.\textit{'' Em sua opini�o, as tr�s principais
necessidades de TI da UFSM campus  Frederico Westphalen s�o:''} entre outras
respostas os entrevistados disseram.
Melhora da internet - Capacita��o sobre o SIE - Cria��o de planilhas/sistemas para 
facilitar o acompanhamento e a avalia��o dos servi�os prestados e da popula��o atendida,
 melhoria na telefonia entre outras respostas.

Na Ultima quest�o: \textit{''De uma forma geral, o grau de satisfa��o que tenho
em rela��o a UFSM campus  Frederico Westphalen com os recursos de TI disponibilizados � condizente  
com minhas necessidades.''} Obteve 59 respostas onde 1 pessoas responderam
concordar totalmente com a quest�o (1,7\%), 36 concordaram (61\%), 6 manifestaram se
indiferentes (10,2\%), 14 discordaram (23,7\%) e 2 discordam totalmente (3,4\%).
 
 \begin{figure}[H]
\centering
\includegraphics[scale=0.7]{figuras/questao8.jpg}
\caption{Gr�fico relacionado a quest�o 1 do question�rio de Governan�a de TI }
\label{figura14}  
\end{figure}
A Figura ~\ref{figura14} mostra resumidamente o resultado d quest�o a cima
citada mostrando no gr�fico de uma forma mais did�tica o resultado d satisfa��o
da comunidade acad�mica em rela��o aos recursos de Ti dispon�veis.

  \section{Trabalhos Relacionados}
\label{sec:trabalhosRelacionados}


  Mancini \cite{Mancini} desenvolveu um estudo de caso em uma institui��o financeira de m�dio 
  porte, descrevendo o processo de implanta��o  Governan�a de TI, 
  analisando documentos e como observa��o participante. A organiza��o financeira utilizou um 
  modelo de boas pr�ticas de Governan�a Corporativa publicado pelo IBGC (Instituto Brasileiro 
  de Governan�a Corporativa). A organiza��o era dividida em tr�s �reas: Desenvolvimento de 
  Sistemas, Suporte T�cnico e Processos, com problemas de imagem da TI ruim, projetos com 
  altos custos e baixo retorno de investimento e n�o alinhamento das estrat�gias de neg�cios 
  com 
  as estrat�gias de TI. Foi contratado uma consultoria especializada para 
  avaliar a organiza��o e definir o n�vel de maturidade, definindo um projeto para se 
  alcan�ar o n�vel 3 de maturidade(Definido) sendo que at� o momento estava no n�vel 2
  (Repetitivo), simplificando processos, monitorando, mensurando resultados e detectando 
  desvios. O auge da pesquisa foi a conquista do n�vel 3 usando framework COBIT, a partir de 
  ent�o se observou mudan�as 
  na cultura da organiza��o. Mesmo seguindo a metodologia o 
  estudo de caso apresenta limita��es, pois em outras organiza��es podem ocorrer de forma 
  diferente. No entanto diferindo do 
  estudo realizado na UFSM uma institui��o de ensino p�blica que foi estudado ferramentas 
  j� existentes na universidade usando como norteador o framework ITIL para analise de tal 
  ferramenta, o estudo relacionado implantou a Governan�a de TI em um institui��o financeira 
  que basicamente busca o  lucro, usando frameworks e boas pr�ticas de Governan�a Corporativa 
  publicado pelo IBGC.


 
	
    Neves \cite{Neves} realizou um estudo de caso numa empresa p�blica, criada para gerir 
    outras empresas federais, onde s�o oferecidos servi�os como consultorias, desenvolvimento, 
    suporte, provimento, treinamento e administra��o. Para coleta de dados usou-se entrevistas,
     an�lise de relat�rios e planilhas, comparando dados obtidos com o framework de boas 
     pr�ticas do ITIL, com a 
     coleta de dados, an�lise e resultados, aplicados no corpo t�cnico, identificando 
     problemas do dia-a-dia. Finalizando realizando-se uma 
     an�lise e amostra aos t�cnicos dos cinco problemas mais frequentes, buscando solu��o. 
     Ap�s foi feito um diagnostico com identifica��o dos problemas, identificando informa��es de falhas existentes e as 
     solu��es, levantando dados num�ricos para quantificar informa��es. Depois de identificados 
     o problema foi, relacionando os cinco mais frequentes 
     e transcrevendo as de forma � facilitar a quantifica��o de informa��es, identificando 
     defici�ncias nos processos de diagn�sticos tendo como 
     escopo a parte de diagn�stico e resolu��o de problemas, que no ITIL � contemplada dentro 
     do controle de problemas. Uma importante diferen�a � que o estudo de caso em quest�o foi 
     desenvolvido em uma institui��o de ensino enquanto que o estudo relacionado numa empresa 
     como fun��o de gerir outras empresas p�blicas enquanto que o estudo relacionado analisou 
     o gerenciamento de problemas o estudo em quest�o na UFSM procurou levantar portf�lios de 
     servi�os e analisar os mesmos para assim fazer um melhor uso de ferramentas j� existentes 
     e sugerir usa de ferramentas de software livre.

 
Morais \cite{Morais} realizou um estudo de caso usando uma abordagem qualitativa, interagindo 
apenas com os principais gestores de TI, essa pesquisa foi de natureza explorat�ria, 
primeiramente realizada uma pesquisa bibliogr�fica para levantamento de informa��es sobre a 
aplica��o de Governan�a de TI, Governan�a Corporativa e seus principais frameworks. 
Realizando uma pesquisa de campo para identifica��o de problemas, coletando informa��es com 
os gerentes da organiza��o, relacionando estas informa��es com os frameworks de boas pr�ticas 
identificando benef�cios e dificuldades, em seguida determinado a��es a serem tomadas e dessa 
forma, alinhar a estrat�gia de TI com a estrat�gia da organiza��o, realizando um plano de a��o 
designando gestores para realizar as a��es na. Ap�s analise de dados, foi constru�do o mapa 
estrat�gico de TI, identificando melhores pratica para cada �rea da organiza��o, visando a 
melhoria na cultura da organiza��o. COBIT e ITIL foram instrumentos viabilizadores orientando 
as a��es para desenvolver instrumentos de avalia��o b�sica e na orienta��o de pr�ticas na 
organiza��o, aprimorando a gest�o estrat�gica da organiza��o. O PMBOK tamb�m fez parte do 
processo de melhoria dando apoio aos lideres do neg�cio. COBIT foi um dos principais 
norteadores neste estudo, enquanto que para o estudo de caso na UFSM, o framework ITIL foi 
o fundamentador da pesquisa disponibilizando suporte para analise e descri��o de dados 
levantados, identificando portf�lio de servi�os.

         
  Masson~\cite{Masson} realizado uma pesquisa na Administra��o P�blica, o objetivo desta 
  pesquisa foi avaliar, junto a �rg�os da Administra��o P�blica Federal, o alinhamento da 
  Governan�a de TI com a Governan�a Corporativa. Esta investiga��o utiliza a abordagem 
  quantitativa. Os dados referentes � Governan�a Corporativa e � de TI e os principais 
  modelos de governan�a foram obtidos em pesquisa documental. Foram aplicados instrumentos de 
  coleta de dados: pesquisa em bases de dados, documentos organizacionais e question�rios 
  semifechados. A pesquisa foi respondida por respons�veis da �rea de TI, ap�s analise foi 
  conclu�do que a Governan�a de TI era pouco ou nada usado como auxilio na tomada de decis�o. 
  Os resultados obtidos na pesquisa, remetem a uma baixa atua��o da Alta Administra��o na 
  Governan�a de TI nas institui��es pesquisadas. O estudo relacionado anteriormente difere-se 
  do realizado na UFSM campus de Frederico Westphalen, por este ter usado como framework boas 
  pr�ticas do ITIL e do ISO/IEC38500 como norteadores para analisar e alinhar TI a ferramenta 
  Help Desk da organiza��o. Enquanto que Masson em seu estudo busca o alinhamento de Governan�a 
  de TI a Governan�a Corporativa.
  
 \cite{raposo}desenvolveu um estudo dos benef�cios de utilizar ferramentas de
 benchmarking  em universidades destacando que as universidades tem que ser entidades que promovam a 
  sustentabilidade. Devido o aumento de custo para manuten��o das mesmas, a
  redu��o de financiamento p�blicos e o aumento de alunos nas universidades.  Dessa 
  forma neste artigo foi demonstrada a import�ncia do benchmarking para as universidades, 
  pois  ajudam na avalia��o e detec��o de problemas organizacionais, desenvolvendo a �rea de 
  governan�a pr� sustent�vel. Tendo como fundamenta��o a melhor tomada de decis�o dando um 
  direcionamento nas mesmas, adequando a universidade na legalidade. No caso deste estudo  
  estava orientado para uma estrat�gia de implementar a cria��o de sistemas de governan�a, 
  baseando no desenvolvimento de projetos simples, mensur�veis, respons�veis, realistas e 
  relacionados as equipes. A abordagem foi baseado na implementa��o de pr�ticas de marketing 
  p�blicos, que s�o baseados na mistura de projetos inteligente com uma abordagem competitiva, 
  controlada e corrigida por meio do uso de ferramentas de benchmarking direcionado para 
  uma implementa��o eficaz.  O benchmarking � um processo de aprendizagem que est�
   estruturado de forma a torna poss�vel avaliar os produtos, servi�os e os pontos fortes, 
   fracos de uma organiza��o, visando o pr�prio aperfei�oamento e auto-controle, um
    processo sistem�tico que mede e compara os processos da organiza��o que trabalha com 
    as demais organiza��es consideradas refer�ncias.  Atrav�s desse processo
    definir papeis e responsabilidades. Fazendo ressalva para a utiliza��o de
    plataformas digitais, al�m disso, estes
plataformas deve ser adequada para a proposta de ferramentas de benchmarking apresentadas e 
aplicada na universidade. Diferido deste estudo o realizado na UFSM foi analisado dados da 
mesma, identificando ferramentas utilizadas, propondo o uso de ferramentas de software 
livre, fazendo uso do framework ITIL para fundamentar a pesquisa tamb�m ISO e o Cobit.
 
  
  


	   

  %\begin{}
%Aqui vai o conclusao\ldots
%\end{conclusoes}
\section{Conclus�es}
\label{sec:conclusoes}
Ap�s estudo realizado foi poss�vel concluir que Governan�a de TI � uma ferramenta 
muito importantes para que organiza��es consigam manter se em pleno funcionamento, 
para isso existem frameworks para auxiliar nessa estrutura��o entre eles os mais usados 
s�o o COBIT, ITIL e ISO/IEC38500, o embasamento te�rico direcionou o estudo e dessa forma 
fez-se compreender que Governan�a de TI esta em pleno desenvolvimento e sua import�ncia � 
cada vez mais vis�vel. Quando observado a utiliza��o de Governan�a de TI na Universidade 
Federal de Santa Maria no campus de Frederico Westphalen viu-se que e muito pouco utilizada.

Com o avan�o da tecnologia e a acelera��o cada vez maior da mesma �  importante o 
acompanhamento de Governan�a de TI em qualquer organiza��o seja ela p�blica ou privada 
de qualquer tamanho, pois d�o suporte a tomada de decis�es, o mais importante � que tudo 
que existir de controle na institui��o pode ser aproveitado n�o � necess�rio que seja 
come�ado do zero para existir Governan�a de TI as ferramentas existentes,
 para dai em diante ser incrementado novos planos de implementa��o, as
 organiza��es precisam ser controladas e para tr controle elas precisam ser
 gerenciadas, por essa resalva que que governan�a e gerencia devem caminhar
 juntas para dessa forma atingir os objetivos desej�veis.
 
Por�m ao estudar Governan�a de TI � encontrada diversas dificuldades entre elas est�o a 
necessidade de investimentos tanto em recursos financeiros quanto em recursos humanos, 
dificultando dessa forma a implanta��o de Governan�a de TI, pois, h� a
necessidade de pessoas treinadas e com tempo dedicado nesse processo, para dessa
forma alinha-la a ao neg�cio. Uma alternativa relevante na quest�o � o uso de
ferramentas livres dispon�veis no software p�blica, neste trabalho voi estudado
a ferramenta CITSMART, est� por sua vez possibilita a cobertura de todas as
�reas da UFSM campus de FW, sendo que a mesma disponibiliza de servi�os bem
completos, por�m a maior entrave ficaria no treinamento de pessoa, sendo que a
mesma por ser ampla demandaria de m�o de obra treinada.

Por Governan�a de TI ser um assunto t�o em alta na atualidade e ser um assunto que cada 
vez mais discutido pelas organiza��es � poss�vel desenvolver estudos, com esse
intuito foi realizado estudo na UFSM  campus de FW analisando a ferramenta Help
Desk do campus verificando que a mesma n�o � institucionalizada, � usada apenas para 
controle interno do setor, foi feito um levantamento de portf�lio de servi�os da mesma 
onde visto que a ferramenta suporta no apenas um servi�o que � a abertura de chamados onde s�o cadastradas as atividades 
realizadas e posteriormente gerados relat�rios. Para poder ser feita est� pesquisa 
anteriormente foi realizado uma pesquisa liter�ria de Governan�a de TI e seus principais 
frameworks, COBIT, ITIL e ISO/IEC38500 para dessa forma embasar a pesquisa e definir assim 
portf�lio a ser seguido, optando por utilizar ITIL e ISO para suportar a pesquisa. 

Governan�a � um assunto muito amplo dessa forma como uma das ferramentas de
estudo foi aplicado um question�rio no campus de FW para dessa form oobter
informa��es onde foi poss�vel observam a amplitude de alcance que a TI tem, por
esse motivo � poosivel afirmar que � necess�rio um maior controle da mesma
alinhando ela a todas as �reas da organiza��o.

Ap�s o estudo da ferramenta Help Desk utilizada na UFSM campus de FW, o
question�rio aplicado com a comunidade acad�mica e a ferramenta CITSMART, �
poss�vel tra�ar os passos futuros, a an�lise de novas ferramentas possibilita
aumentar o conhecimento e assim ter uma perspectiva de quanto Governan�a de TI �
importante em uma organiza��o. A coleta de novos dados e a institucionaliza��o
da ferramenta Help Desk, que at� momento � utilizada somente para controle
interno do setor de TI da universidade, poderia ser um come�o pra o alinhamento de TI
com as demais �reas da universidade.

 
 
 %%%-----------------------------------%% END ARTIGO
 
%%% Referencias bibliograficas no padrao da SBC
\bibliographystyle{sbc}

%%% Referencias bibliograficas no padrao da IEEE
%\bibliographystyle{IEEEtran} 
 
 %%% Nome do arquivo .bib contendo as suas referencias bibliograficas
\bibliography{referencias}

 
\end{document}
